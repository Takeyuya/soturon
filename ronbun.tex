\documentclass[a4paper,9pt,onecolumn]{ltjsarticle} % ← lualatex ではなく ltjsarticle
\usepackage[ipaex]{luatexja-preset}  % 日本語フォントの手当て(IPAex)
\usepackage{ifthen}
\usepackage{algorithm}
\usepackage{algpseudocode}
\usepackage{amsmath}
\usepackage{graphicx}
\usepackage{here}
\usepackage{algpseudocode}
\usepackage{titlesec}

\makeatletter
\renewcommand{\maketitle}{%
\ifthenelse{\equal{\course}{bachelor}}{%
  \begin{titlepage}
    \vspace*{10mm}
    \begin{center}
      {\huge\@gyear 年度\hspace{5mm}\@thesis\par}
      \vfill
      {\huge\bfseries\@title\par}
      \vfill
      \vspace{10mm}
      {\Large\@date\par}
      \vspace{20mm}
      {\Large\@department\@departmentsfx\par}
      {\Large (学生番号: \@studentid)\par}
      \vspace{5mm}
      {\LARGE\@author\par}
      \vspace{55mm}
      {\Large 和歌山大学\@faculty}
      \vspace{10mm}
    \end{center}
  \end{titlepage}}{}%
\ifthenelse{\equal{\course}{master}}{%
  \begin{titlepage}
    \begin{center}
      \vspace*{8mm}
      {\Large\@gyear 年度\hspace{5mm}\@thesis\par}
      \vfill
      {\huge\bfseries\@title\par}
      \vfill
      {\Large \@date\par}
      \vspace{40mm}
      {\Large 和歌山大学\@faculty\par}
      \vspace{27mm}
      {\Large 学生番号: \@studentid\par}
      {\LARGE \@author\par}
      \vspace*{28mm}
    \end{center}
  \end{titlepage}%
  \begin{titlepage}
    \begin{center}
      \vspace*{8mm}
      {\huge\@etitle\par}
      \vspace{12mm}
      {\Large by\par}
      \vspace{9mm}
      {\huge\@eauthor\par}
      \vspace{27mm}
      {\LARGE\@ethesis\par}
      \vfill
      {\Large \@efaculty\par}
      \vspace{4mm}
      {\Large Wakayama University\par}
      \vspace{23mm}
      {\Large \@edate\par}
      \vspace*{14mm}
    \end{center}
  \end{titlepage}}{}%
}

%% 概要
\def\abstract{\newpage\pagenumbering{roman}\section*{概 要}}
\def\endabstract{}
%% 目次
\def\tableofcontents{
  \newpage
  \section*{目 次\@mkboth{目 次}{目 次}}
  \@starttoc{toc}}
%% 図目次
\def\listoffigures{%
  \newpage
  \section*{図 目 次\@mkboth{図 目 次}{図 目 次}}
  \@starttoc{lof}}
%% 表目次
\def\listoftables{%
  \newpage
  \section*{表 目 次\@mkboth{表 目 次}{表 目 次}}
  \@starttoc{lot}}
%% 謝辞
\def\acknowledgements{\newpage\section*{謝 辞}}
\def\endacknowledgements{}
%% 参考文献
\def\thebibliography#1{%
  \newpage
  \section*{参 考 文 献\@mkboth{参 考 文 献}{参 考 文 献}}
  \list{[\arabic{enumi}]}
    {\settowidth{\labelwidth}{[#1]}
     \leftmargin=\labelwidth
     \advance\leftmargin by \labelsep
     \usecounter{enumi}}
  \def\newblock{\hskip .11em plus .33em minus .07em}
  \sloppy
  \clubpenalty=4000 \widowpenalty=4000
  \sfcode`\.=1000\relax}
\let\endthebibliography=\endlist
\makeatother

\makeatletter
\def\course{bachelor}
\def\@gyear{2025}
\def\@thesis{卒業論文}
\def\@title{DIDに基づいたIoTデータ管理システムの構築と評価}
\def\@date{2026年2月10日}
\def\@department{システム工学部}
\def\@departmentsfx{システム工学科}
\def\@studentid{60276128}
\def\@author{竹内 結哉}
\def\@faculty{システム工学部}
\makeatother

\newcommand{\sectionbreak}{\clearpage}

\begin{document}
\maketitle
\tableofcontents

\section{はじめに}
近年,IoT機器の爆発的な増加に伴い,生成されるデータ量は急激に増加している.
さらに,IoTは家庭や産業,医療,農業など多様な分野で活用されるようになり,生成されるデータの種類や粒度も一層多様化している.

また,従来の中央集権型によるIoTデータ管理には主に次の3つの課題が存在する.

第一に,スケーラビリティの問題である.
IoTデバイスの急増により,そのデータを保存する中央サーバーへの負荷が指数関数的に増大し,処理能力の限界に達する可能性がある.

第二に,セキュリティ上の問題である.中央サーバーは単一障害点となりやすく,攻撃対象として脆弱である.

第三に,プライバシー保護の問題である.個人情報を含むIoTデータが集中することで,情報漏洩時の被害が甚大化するリスクがある.

これらの課題を解決するために,本研究ではユーザ主権型IDに基づいた分散型データ管理システムの実現を目指す.
本システムは,以下の三点を重視して設計されている.

\begin{enumerate}
\item データの分散管理:中央集権型管理から脱却し,分散型ファイルシステムであるInterPlanetary File System(以下IPFS)を用いることで,
      単一障害点を排除しシステムの堅牢性を向上させる.
\item ユーザの真正性確保:分散管理環境におけるなりすまし防止のため,分散型識別子であるDecentralized identifier(以下DID)を活用し,
      データ所有者の身元を保証する.
\item データの信頼性と改ざん防止:ブロックチェーンを活用し,データが改ざんされていないことを検証可能とする.
\end{enumerate}
以上の要素を組み合わせることで,IoTデータに対する分散型かつ信頼可能な管理基盤を構築することを目指す.

\section{関連研究}
IoTデータ管理におけるプライバシー確保は,従来より大きな研究課題とされている.
現行のIoTシステムは,多くが中央集権型のクライアントサーバーモデルに依存しており,生成される膨大なデータはサービスプロバイダを介して管理される.
このような中央集権型モデルは,ユーザの行動履歴や個人情報が第三者に漏洩・不正利用されるリスクを内包している.
この問題に対処するため,近年ではブロックチェーンを基盤とした分散型データ管理アーキテクチャの研究が進められている.

IoTとブロックチェーンの統合に関する典型的なユースケースとしては,
(1) イベントの改ざん防止ログ,
(2) アクセス制御の管理,
(3) IoTセンサーデータの購入
などが挙げられる\cite{cite2}.
これらの研究では,ブロックチェーンが持つ分散性と改ざん耐性を活用し,IoT環境におけるデータ完全性の確保を目指している.
一方で,既存の研究の多くは概念設計やプロトコル提案に留まり,Proof-of-Conceptレベルの実装や性能評価が十分に行われていないことが課題とされている.

Aliら\cite{cite1}は,ブロックチェーンとIPFSを組み合わせた\textbf{モジュラーコンソーシアムアーキテクチャ}を提案している.
このモデルでは,IoTデバイスをプライベートな「サイドチェーン」にグループ化し,アクセス制御の管理を「コンソーシアムブロックチェーン」によって実現する.
サイドチェーンはセンサーデータ生成イベントを記録し,コンソーシアムブロックチェーンはアクセス要求の不変なログを保持することで,
プライバシー保護とアカウンタビリティを両立させている.
さらに,データ自体はIPFS上に保存され,ブロックチェーンはハッシュのみを記録することで,ストレージ効率とセキュリティの両立を実現している.

評価実験として,Ethereum(PoW)およびMonax(PoS/Tendermint)を用いた性能比較が行われている.
その結果,Monaxはサイドチェーンレベルで低い処理オーバーヘッドを示す一方で,コンソーシアムレベルでは高いネットワークトラフィックオーバーヘッドが課題となった.
Ethereumはコンソーシアムレベルで通信効率に優れるものの,PoWに基づく高い計算コストが問題点として指摘されている.
このように,ブロックチェーンのコンセンサスメカニズムの選択は,IoT分散アーキテクチャの実用性に直接影響を与えることが明らかとなっている.

また,\cite{cite2}では,IoTセンサーから得られるデータを\textbf{二重チャネル(Dual Channel)}で配信するミドルウェアが提案されている.
一方はEthereumとIPFSを用いたインテグリティチャネルで,データ完全性を保証する.
もう一方はMQTTを利用したリアルタイムチャネルで,低遅延かつ高速な配信を実現する.
この方式は,後からインテグリティチャネルを参照することで,リアルタイムチャネル経由で受け取ったデータの改ざん検出が可能になる点に特徴がある.
ただし,Raspberry Piを用いた評価では,インテグリティチャネルにおいて数百秒規模の遅延やデータ欠損が発生するなど,
ブロックチェーンのオーバーヘッドとリソース制約の影響が確認されており,実運用に向けた課題も示されている.

本研究は,これらの先行研究の知見を基盤としつつ,IoTデータ管理における\textbf{DID}と\textbf{VC}の統合に焦点を当てる.
既存研究が主に公開鍵基盤に依存したアクセス制御を行っていたのに対し,本研究ではDID/VCを導入することで,
より柔軟かつ標準化された認証・検証モデルを提供する点に新規性がある.
また,EthereumおよびIPFSを用いたローカル環境での実装と性能測定を通じて,スケーラビリティと実用性の両面から評価を行うことを目的としている.

\section{準備}
本研究では,分散型データ管理の基盤技術としてIPFS,ブロックチェーンおよびDIDを用いる.
本章では,これらの技術の概要を説明し,さらに本研究で使用した実験環境について述べる.

\subsection{InterPlanetary File System(IPFS)}
IPFSは世界中のコンピュータ(ノード)に分散的にデータを保存するP2P型のファイルシステムである.
IPFSでは中央集権的なサーバを介さず,データを分散的に管理しており,耐故障性,負荷分散,耐検閲性,改ざん耐性に優れている.
特徴は「コンテンツアドレス方式」である点で,保存されたファイルはその内容を基に計算されるContent Identifier(以下CID)によって参照される.
CIDはファイル内容のハッシュ値であるため,以下の性質を持つ.

\begin{itemize}
  \item 同一内容のファイルは必ず同じCIDとなる.
  \item 1ビットでも内容が変更されれば別のCIDとなる.
  \item CIDから元のデータを推測することはできない.
\end{itemize}
この仕組みにより,ファイルが改ざんされていないかをCIDの比較によって確認できるため,改ざん耐性に優れる.

本研究では,ユーザが保有するIoTデータをIPFSに保存し,得られたCIDをブロックチェーンに記録することで,データの真正性と参照可能性を確保している.

\subsection{ブロックチェーン}
ブロックチェーンは,ネットワーク上の複数のノードが同一のデータを共有し,合意形成に基づいて取引履歴を記録する分散型台帳技術である.
記録されるデータは複数の取引をまとめた「ブロック」に格納され,各ブロックは直前のブロックのハッシュ値を保持することで鎖状に連結される.
この構造により,一部のブロックが改ざんされると以降すべてのブロックの整合性が崩れるため,改ざんは即座に検知される.

特に重要なのは,ハッシュ値の性質である.
ハッシュ値はデータから一方向的に算出される識別子であり,内容にわずかな変更があっても全く別の値となる.
また,ハッシュ値から元のデータを復元することはできない.
ブロックチェーンではこの性質を利用し,データの完全性を保証している.

本研究では,ユーザAが発行したDIDとDID Documentをブロックチェーンに登録し,IoTデータの所有者であることを証明するための基礎情報として利用する.

さらに本研究のシステムでは,発行者がユーザに対してVerifiable Credential(以下VC)を発行し,
IoTデータが正当なデバイスによって生成されたものであることを保証する仕組みを構築する.
VCの検証時には,DID Documentに記録された公開鍵により署名を確認し,データの真正性を確認することができる.

また,IoTデータの要約情報(CID)およびDID Documentをスマートコントラクト経由でブロックチェーンへ記録することで,
データ登録の証跡を改ざん不能に保持できるようにしている.

さらに,既存研究においても議論されているように,Proof-of-Work(以下PoW)はエネルギー消費が大きく,
リソース制約のあるIoT環境には適さないことが指摘されている\cite{cite2}.
そのため,本研究においてもPoWベースのブロックチェーンは採用せず,より実装や評価に適したEthereum環境を主に使用することとした.

\subsection{Decentralized identifier(DID)}
DIDは特定の中央管理者に依存せずに個人が自分自身で生成・管理できる識別子であり,分散型デジタルアイデンティティの基盤となる技術である.
DIDはそのDIDに紐づく公開メタデータであるDID Documentが存在している必要がある.
DID Documentには識別子の情報やDIDの所有者が使用する公開鍵の情報が含まれており,DIDの正当性や鍵の正しさを検証する場面ではDIDに紐づいたDID Documentを取得して使用される.
DIDからDID Documentを取得するという処理をDID解決と呼ぶ.
DID Documentはブロックチェーンなどの改ざん耐性を持つ基盤に保存されることで,第三者がその内容を検証可能となる.

\section{システム構成}
本研究で提案するシステムは,IoTデータを分散的に管理するために,IPFS,ブロックチェーンおよびDIDを連携させたものである.
本章では,まずシステム全体の流れを示した後,各要素の役割について説明する.

\subsection{システムの概要}
本研究で提案するシステムの全体像を図\ref{fig:system-overview}に示す.
本研究で構築したシステムは,IoTデータを提供するユーザA,データの真正性を確認する発行者,および最終的にVCを検証する検証者により構成される.
図\ref{fig:system-overview}に示すように,ユーザAは自身のIoTデータをIPFSに格納し,その結果として得られるCIDを,
自身の識別子であるDIDとともにブロックチェーンに記録する.
その後,発行者がIoTデータの真正性を保証するVerifiable Credential(以下VC)を発行し,
検証者がブロックチェーン上のDID情報とVCを突き合わせることで,ユーザAの正当性を検証する.

\begin{figure}[t]
  \centering
  \includegraphics[width=0.9\linewidth]{figures/figure1.png}
  \caption{提案システムの概要図}
  \label{fig:system-overview}
\end{figure}

\subsubsection{データ所有者(ユーザA)}
ユーザAは,自身の識別子としてDIDを保持し,DID Documentをブロックチェーンに格納する.
さらに,保持しているIoTデータをIPFSに格納し,IPFSから返されるCIDを取得する.
ユーザAは,CIDと自身のDIDを組み合わせてブロックチェーンに登録することで,データの所有者であることを保証する.

\subsubsection{発行者(企業など)}
発行者とは,ユーザAに対してVCを発行する主体である.
本研究ではユーザAが所持しているIoT機器の製造元企業などを想定している.
発行者はユーザAが保持しているIoTデータが自社製品によって生成されたデータであることを確認し,その真正性を保証するVCをユーザAに発行する.

\subsubsection{検証者}
検証者は,ユーザAとIoTデータを取引する相手,すなわちデータの受領者を想定している.
検証者は,ユーザAから提示されたVCを受け取り,ブロックチェーン上のDID Documentと照合することで,ユーザAが真正なデータ所有者であることや,
IoTデータについて企業が保証していることについて確認する.

\section{システム全体のフロー}
システムの処理は,\textbf{登録フェーズ}と\textbf{検証フェーズ}の2つに大別される.

登録フェーズではDID・IoTデータがブロックチェーンおよびIPFSに記録され,企業によるVC発行までが行われる.
一方,検証フェーズでは,提示されたVCの署名検証を通じてユーザAが真正なデータ提供者であることを確認し,安全なデータ取引を可能にする.

以下では,両フェーズの詳細な処理について説明する.

\subsection{登録フェーズ}

登録フェーズのフローチャートを図\ref{fig:system_flowchart_regist}に示す.
本フェーズはDIDの生成・登録,IoTデータの保存,CIDの登録,および発行者によるVC発行,ユーザAによるVCへの署名までの流れで構成される.

\subsubsection{DIDの生成とDID Documentの登録}

まず,ユーザAおよび発行者はそれぞれDIDを生成し,それに対応するDID Documentを作成する.
作成したDID Documentはブロックチェーンへ登録される.
DID Documentのブロックチェーンへの登録手順はスマートコントラクト\texttt{registerDIDDocument()}(algorithm~\ref{alg:register_did})で規定される.

これにより,各主体の公開鍵が台帳上に保持され,後続処理における署名検証の基盤が整備される.

\subsubsection{IoTデータの保存とCIDの取得}

次に,ユーザAは自身が保有するIoTデータをIPFSに保存する.
保存されたデータに対し,IPFSは一意のCIDを返却する.

\subsubsection{CIDとDIDのブロックチェーンへの記録}

ユーザAは取得したCIDと自身のDIDをブロックチェーンへ登録する.
CIDとDIDのブロックチェーンへの登録はスマートコントラクト\texttt{registerIoTData()}(algorithm~\ref{alg:register_iot})で規定される.
これにより,「どのDIDがどのIoTデータの所有者であるか」が改ざん耐性を持って記録され,第三者は所有者を検証可能となる.

\subsubsection{発行者によるデータ真正性の確認とVC発行}

発行者は,ブロックチェーン上のDIDとCIDの整合性を確認し,ユーザAが登録したデータが真正であるかを検証する.
正当性が確認された場合,企業はユーザAに対してVCを発行する.
VCには発行者のDIDによる署名が付与され,内容の真正性が保証される.

\subsubsection{ユーザAによるVCへの追加署名}

最後に,ユーザAは自身のDIDに紐づく秘密鍵を用いてVCに追加署名を行う.
これにより,企業とユーザAの双方が署名したVCが完成し,検証フェーズにおいて提示・検証可能な証明情報となる.

\begin{figure}[H]
  \centering
  \includegraphics[height=0.9\textheight, keepaspectratio]{figures/flowchart_regist.png}
  \caption{登録フェーズのフローチャート}
  \label{fig:system_flowchart_regist}
\end{figure}

\subsection{検証フェーズ}

検証フェーズのフローチャートを図\ref{fig:system_flowchart_validate}に示す.
本フェーズでは,ユーザAが提示したVCに含まれる署名を順に検証し,データ提供者の正当性を確認する.

\subsubsection{VCの提示と受領}

ユーザAは完成したVCを検証者へ提示する.
検証者はVCを受け取り,内容に含まれる署名情報とDID情報を確認する.

\subsubsection{企業の署名検証}

検証者はまずVCに含まれる発行者の署名を検証する.
署名の公開鍵は,ブロックチェーンへ登録された企業のDID Documentから取得される.
署名が不一致であった場合,VCは不正と判断され処理を終了する.

\subsubsection{ユーザAの署名検証}

企業の署名が正当であった場合,次にユーザAの署名が検証される.
ユーザAの公開鍵は同様にDID Documentから取得され,署`名が一致する場合,ユーザAがIoTデータの正当な保有者であることが確認される.
不一致の場合,処理は終了する.

\subsubsection{データの取引の実行}

企業およびユーザAの署名がいずれも正当であれば,検証者はユーザAが真正なデータ提供者であると判断できる.
この確認を基に,検証者はユーザAと安全にデータ取引を実行する.

\begin{figure}[H]
  \centering
  \includegraphics[height=0.9\textheight, keepaspectratio]{figures/flowchart_valid.png}
  \caption{検証フェーズのフローチャート}
  \label{fig:system_flowchart_validate}
\end{figure}

\subsection{スマートコントラクトによる登録処理}

本研究で使用したスマートコントラクトでは,ユーザのDID DocumentおよびIoTデータ(CID)をブロックチェーンに記録するために\texttt{registerDIDDocument}および\texttt{registerIoTData}の2つの関数を提供している.
それぞれの処理内容を擬似コードとして以下に示す.

\begin{algorithm}
\caption{DID Documentの登録処理(registerDIDDocument)}
\label{alg:register_did}
\begin{algorithmic}[1]
\Require DID, DID Document(JSON形式)
\State 呼び出し元アドレスを取得する(これをユーザ識別子として扱う)
\State DID Documentを以下の形式で保存する:
\State \hspace{3mm} \texttt{records[呼び出し元アドレス].append( (DID, DID\_Document) )}
\State DID登録イベントを発行する
\end{algorithmic}
\end{algorithm}

\begin{algorithm}[H]
\caption{IoTデータ(CID)の登録処理(registerIoTData)}
\label{alg:register_iot}
\begin{algorithmic}[1]
\Require DID, CID(IPFSで取得したハッシュ値)
\State 呼び出し元アドレスを取得する
\State IoTデータを以下の形式で保存する:
\State \hspace{3mm} \texttt{IotData[呼び出し元アドレス].append( (DID, CID) )}
\State IoTデータ登録イベントを発行する
\end{algorithmic}
\end{algorithm}

\section{実験と考察}
本章では,本研究で構築したDID・VC・IPFS・Ethereumを用いた分散型データ管理システムについて,機能動作試験および性能評価を通してその有効性を検証する.

\subsection{実験目的}
本研究の実験目的は,大きく2つである.
1つ目は提案システムが問題なく動作することを検証することである.
具体的には,DIDの生成からDID Documentの登録,IoTデータのIPFSへの保存,取得したCIDのブロックチェーンへの保存,発行者によるVCの発行,
および検証者によるVCの検証に至るまでの一連の処理を実装し,それらが想定通り正しく実行されることを確認する.
これにより,DID・VC・IPFS・ブロックチェーンを統合した提案システムが,データとその所有者の正当性を一貫して保証できることを検証する.

2つ目は,提案システムの性能を定量的に評価し,DIDおよびVCを統合したことがシステム全体の性能に与える影響を明らかにすることである.
特に,IPFSおよびブロックチェーンを用いたIoTデータ管理に関する先行研究が多数存在することを踏まえ,これらを基準的な処理として位置づけたうえで,
DIDおよびVCに関連する処理が新たな性能上のボトルネックとなり得るかどうかに着目して評価を行う.

\subsection{実験環境}
実験は以下の環境で行った.
\begin{itemize}
  \item OS:Windows 11 Home
  \item CPU:AMD Ryzen 5 PRO 7530U with Radeon Graphics(2.00Hz)
  \item メモリ:16GB
  \item IPFS:go-ipfs v0.35.0
  \item ブロックチェーン環境:Ganache v7.9.2
  \item Solidity: v0.5.16
  \item Node.js: v18.20.7
  \item Truffle: v5.11.5
  \item Web3.js: v1.10.0
\end{itemize}

\subsection{実験方法}
本節では,本研究で構築したシステムに対して実施した実験方法を述べる.
実験は大きく次の2段階に分けて行った.

\begin{enumerate}
  \item \textbf{機能動作試験}
  DID Documentの登録,IoTデータのIPFSへの保存,CIDのブロックチェーンへの保存,発行者によるVC発行,検証者によるVC検証という一連の処理が正しく実行されることを確認した.

  \item \textbf{性能評価}
  システム内の主要な処理について処理時間を計測し,応答時間を評価した.
  測定対象は以下のとおりである.

  \begin{itemize}
    \item IoTデータのIPFSアップロード時間
    \item ブロックチェーンへのCID登録時間
    \item VC発行処理時間
    \item VC検証処理時間
  \end{itemize}
\end{enumerate}

なお,本研究における「データ1件」とは,センサから取得された1時点分のIoTデータを表すJSONオブジェクト1つ分を指す.
具体的には,タイムスタンプ,温度,湿度,照度を含む以下の形式のJSONデータ1つを1件とする擬似データを使用している.

\begin{verbatim}
  {
    "timestamp": "2025-12-14 18:32:58",
    "temperature": 21.65,
    "humidity": 63.41,
    "light": 926
  }
\end{verbatim}

したがって,データ数10件,100件,1\,000件とは,上記形式のJSONデータをそれぞれ10個,100個,1\,000個まとめたファイルを対象として測定を行ったことを意味する.

\subsection{実験結果}

\subsubsection{機能動作試験の結果}
本研究で構築したシステムが想定した一連の処理を正しく実行できるかを確認した.


\begin{figure}[H]
  \centering
  \includegraphics[width=0.95\linewidth]{figures/register_did.png}
  \caption{DIDおよびDID Documentの生成・登録処理の実行結果}
  \label{fig:register_did}
\end{figure}

\begin{figure}[H]
  \centering
  \includegraphics[width=0.95\linewidth]{figures/ipfs_upload.png}
  \caption{IPFSへのデータアップロードの実行結果}
  \label{fig:ipfs_upload}
\end{figure}

\begin{figure}[H]
  \centering
  \includegraphics[width=0.95\linewidth]{figures/register_cid_to_bc.png}
  \caption{CIDおよびDIDのブロックチェーンへの登録処理}
  \label{fig:register_cid_to_bc}
\end{figure}

\begin{figure}[H]
  \centering
  \includegraphics[width=0.95\linewidth]{figures/issue_vc.png}
  \caption{発行者がVCを発行する処理}
  \label{fig:issue_vc}
\end{figure}

\begin{figure}[H]
  \centering
  \includegraphics[width=0.95\linewidth]{figures/sign_vc_by_userA.png}
  \caption{UserAがVCに署名する処理}
  \label{fig:sign_vc_by_userA}
\end{figure}

\begin{figure}[H]
  \centering
  \includegraphics[width=0.95\linewidth]{figures/validate_vc.png}
  \caption{検証者がVCを検証する処理}
  \label{fig:validate_vc}
\end{figure}

以上より,本研究で提案するデータとその所有者の正当性を保証するフローが一貫して成立することを確認した.

\subsubsection{性能評価の結果}

本研究では,提案システムの実運用を想定した際に,DIDおよびVCを統合したことが性能上のボトルネックとなり得るかを明らかにすることを目的として,
各処理に要する時間の計測を行った.

性能測定は,各処理は20回繰り返し実行した際の平均処理時間を算出することで行った.
まず,IPFSへのIoTデータアップロード処理の性能を評価した.
その結果,データ数10件(931bytes)から1\,000件(96\,793bytes)までのいずれの場合においても,
平均処理時間は約19~21msで推移しており,データサイズの増加に対して大きな遅延は確認されなかった.
また,スループットはデータ数の増加に伴い向上しており,IPFSがIoTデータの集約保存に対して十分な性能を有することが確認できた.

次に,ブロックチェーンへのCID登録処理に要する時間を計測した.
本処理では,データ内容に依存しない条件下において,平均処理時間は63.60msであり,トランザクション処理として一定の時間を要するものの,
安定した性能を示した.

これらの処理と比較し検討するため,VCの発行処理および検証処理に要する時間を計測した.


まず,VCの発行に要する時間を計測した結果,平均処理時間は10.45msであり,
IPFSアップロード処理およびブロックチェーン登録処理と比較しても短い処理時間であった.
このことから,VCの発行処理がシステム全体の性能に与える影響は小さく,性能上のボトルネックとなる可能性は低いと考えられる.

一方,VCの検証処理に要する平均時間は155.78msであり他の処理と比較して相対的に長い処理時間を要する結果となった.

以上の性能評価結果を表\ref{tab:performance_results}にまとめる.

\begin{table}[h]
  \centering
  \caption{各処理における性能評価結果}
  \label{tab:performance_results}
  \begin{tabular}{l|l|r}
    \hline
    処理内容 & 条件 & 平均処理時間 \\ \hline
    IPFSアップロード & 10件(931bytes) & 19.11ms \\
    IPFSアップロード & 100件(9\,279bytes) & 20.78ms \\
    IPFSアップロード & 1\,000件(96\,793bytes) & 21.20ms \\
    ブロックチェーン登録 & --- & 63.60ms \\
    VC発行 & --- & 10.45ms \\
    VC検証 & --- & 155.78ms \\
    \hline
  \end{tabular}
\end{table}


\subsection{考察}
本章では,前章で示した実験結果を踏まえ,提案システムにおいてDIDおよびVCを統合したことがシステム全体の性能に与える影響について考察する.

\subsubsection{DID/VCが性能に与える影響}
まず,VCの発行処理に着目する.
性能評価の結果より,IPFSへのIoTデータアップロード時間は約19~21msで推移し,ブロックチェーンへのCID登録処理は平均63.60msを要することが確認されている.
これらの処理と比較すると,VCの発行処理に要する平均時間は10.45msと短く,システム全体の処理性能に与える影響は小さい.
このことから,VCの発行処理が提案システムの動作において性能上のボトルネックとなる可能性は低いと考えられる.

次に,VCの検証処理に着目する.
VCの検証にかかる処理は155.78msであり,IPFSへのアップロード処理やブロックチェーンへのCID登録処理と比較して,
相対的に長い処理時間を要する結果となった.

しかしながら,VCの検証処理は,IoTデータの保存や更新といった高頻度に実行される処理とは異なりデータの取引時といった特定のタイミングでの実行を想定している.
そのため,通常の運用形態においてはシステム全体の性能に与える影響は限定的であると考えられる一方,
検証処理の実行頻度が高くなる場合には,性能上のボトルネックとなる可能性も否定できない.

\subsubsection{今後の課題}
本稿で提案したシステムは,中央集権的な管理主体に依存せず,分散的にIoTデータの管理を行うことを目的として設計した.
しかしながら,ユーザが保持するデータの正当性を保証する点においては,VCの発行主体として企業などの組織に依存せざるを得ず,
この点は部分的に中央集権的な管理構造となっている.

したがって,完全に分散化されたデータ管理システムの実現という観点からは,データの正当性保証を特定の組織に依存しない仕組みについてさらなる検討が必要であると考えられる.

\section{まとめ}
ここはまとめの章です.

\bibliographystyle{junsrt}
\bibliography{references}
\end{document}
