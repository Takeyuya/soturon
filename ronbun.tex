\documentclass[a4paper,9pt,twocolumn]{ltjsarticle} % ← lualatex ではなく ltjsarticle
\usepackage[ipaex]{luatexja-preset}  % 日本語フォントの手当て(IPAex)
\usepackage{ifthen}
\usepackage{algorithm}
\usepackage{algorithmicx}
\usepackage{algpseudocode}
\usepackage{amsmath}

\makeatletter
\renewcommand{\maketitle}{%
\ifthenelse{\equal{\course}{bachelor}}{%
  \begin{titlepage}
    \vspace*{10mm}
    \begin{center}
      {\huge\@gyear 年度\hspace{5mm}\@thesis\par}
      \vfill
      {\huge\bfseries\@title\par}
      \vfill
      \vspace{10mm}
      {\Large\@date\par}
      \vspace{20mm}
      {\Large\@department\@departmentsfx\par}
      {\Large (学生番号: \@studentid)\par}
      \vspace{5mm}
      {\LARGE\@author\par}
      \vspace{55mm}
      {\Large 和歌山大学\@faculty}
      \vspace{10mm}
    \end{center}
  \end{titlepage}}{}%
\ifthenelse{\equal{\course}{master}}{%
  \begin{titlepage}
    \begin{center}
      \vspace*{8mm}
      {\Large\@gyear 年度\hspace{5mm}\@thesis\par}
      \vfill
      {\huge\bfseries\@title\par}
      \vfill
      {\Large \@date\par}
      \vspace{40mm}
      {\Large 和歌山大学\@faculty\par}
      \vspace{27mm}
      {\Large 学生番号: \@studentid\par}
      {\LARGE \@author\par}
      \vspace*{28mm}
    \end{center}
  \end{titlepage}%
  \begin{titlepage}
    \begin{center}
      \vspace*{8mm}
      {\huge\@etitle\par}
      \vspace{12mm}
      {\Large by\par}
      \vspace{9mm}
      {\huge\@eauthor\par}
      \vspace{27mm}
      {\LARGE\@ethesis\par}
      \vfill
      {\Large \@efaculty\par}
      \vspace{4mm}
      {\Large Wakayama University\par}
      \vspace{23mm}
      {\Large \@edate\par}
      \vspace*{14mm}
    \end{center}
  \end{titlepage}}{}%
}

%% 概要
\def\abstract{\newpage\pagenumbering{roman}\section*{概 要}}
\def\endabstract{}
%% 目次
\def\tableofcontents{%
  \newpage
  \section*{目 次\@mkboth{目 次}{目 次}}
  \@starttoc{toc}}
%% 図目次
\def\listoffigures{%
  \newpage
  \section*{図 目 次\@mkboth{図 目 次}{図 目 次}}
  \@starttoc{lof}}
%% 表目次
\def\listoftables{%
  \newpage
  \section*{表 目 次\@mkboth{表 目 次}{表 目 次}}
  \@starttoc{lot}}
%% 謝辞
\def\acknowledgements{\newpage\section*{謝 辞}}
\def\endacknowledgements{}
%% 参考文献
\def\thebibliography#1{%
  \newpage
  \section*{参 考 文 献\@mkboth{参 考 文 献}{参 考 文 献}}
  \list{[\arabic{enumi}]}
    {\settowidth{\labelwidth}{[#1]}
     \leftmargin=\labelwidth
     \advance\leftmargin by \labelsep
     \usecounter{enumi}}
  \def\newblock{\hskip .11em plus .33em minus .07em}
  \sloppy
  \clubpenalty=4000 \widowpenalty=4000
  \sfcode`\.=1000\relax}
\let\endthebibliography=\endlist
\makeatother

\makeatletter
\def\course{bachelor}
\def\@gyear{2025}
\def\@thesis{卒業論文}
\def\@title{DIDに基づいたIoTデータ管理システムの構築と評価}
\def\@date{2026年2月10日}
\def\@department{システム工学部}
\def\@departmentsfx{システム工学科}
\def\@studentid{60276128}
\def\@author{竹内 結哉}
\def\@faculty{システム工学部}
\makeatother

\begin{document}
\maketitle

\section{はじめに}

\section{背景技術}
\subsection{分散型識別子(Decentralized Identifier:DID)}
従来のインターネットにおける識別子(例:メールアドレスやSNSアカウント)は,特定の企業や組織が発行・管理する中央集権的な仕組みに依存している.
このため,アカウント停止やセキュリティ侵害のリスク,さらにはユーザーが自らのアイデンティティを完全に管理できないという課題が存在する.

分散型識別子(以下DID)は,この問題を解決するためにW3Cにより標準化が進められている新しい識別子である.
DIDは「did:example:xxxx」のような形式を持ち,ブロックチェーンなどの分散型台帳に基づいて管理される.
各DIDには対応するDID Documentが存在し,公開鍵や認証手段,サービスエンドポイントなどを含むことで,所有者の真正性を保証する.

本研究においては,DIDを用いることでIoTデータ所有者の識別と認証を分散的に行い,ユーザー自身が自己主権的にデータを管理できる仕組みを実現する.
これにより,なりすましの防止やデータ所有権の正当性証明が可能となる.

\subsection{分散型ファイルシステム(InterPlanetary File System:IPFS)}
IoT機器から生成されるデータは膨大かつ多様であり,従来の中央集権型サーバに保存する方式では,スケーラビリティの限界,単一障害点,セキュリティリスクなどの問題が生じる.
特に,中央サーバーにデータが集中すると攻撃対象となりやすく,情報漏洩時の被害も大きくなる.

惑星間ファイルシステム(以下IPFS)はこれらの課題を解決するために提案された分散型のファイルシステムである.
IPFSはコンテンツアドレス方式を採用しており,ファイルはその内容に基づくハッシュ値で一意に識別される.
このため,データが改ざんされれば異なるハッシュ値となり,完全性の検証が容易である.
また,ピアツーピアネットワークを介して効率的にデータを配信できるため,冗長性・可用性に優れ,単一障害点を排除できる.

本研究では,IPFSを利用してIoTデータを分散的に保存し,そのハッシュ値のみをブロックチェーンに記録する方式を採用する.
これにより,ブロックチェーンにデータ本体を保存する必要がなくなり,処理負荷を軽減しつつデータの信頼性を保証できる.


\subsection{DIDとIPFSの統合による効果}
DIDとIPFSを組み合わせることで,データ管理における2つの要件,すなわち「所有者の正当性」と「データの改ざん防止」を同時に満たすことが可能となる.
具体的には,IoTデータをIFPSに保存し,そのハッシュ値をDIDとともにブロックチェーンに記録することで,
\begin{itemize}
  \item DIDによるデータ所有者の真正性の保証
  \item IPFSハッシュによるデータ完全性の保証
\end{itemize}
を実現できる.
これにより,中央集権型管理の問題を解消し,信頼可能でユーザー主権的なIoTデータ流通基盤の構築を可能とする.

\end{document}
